\chapter{Conclusion}
In this dissertation, the work on two projects are covered.

In the Three Dimensional Edwards-Anderson Model project, we developed an efficient
GPU implementation of Monte Carlo simulation with parallel tempering and
multispin coding technique. We achieved world-leading performance in GPU 
implementation on this model. We then used the code to study the model in an 
external field. Our results show that susceptibilities are not normally distributed and mean is 
dominated by rare events. As a result, a huge number of disordered samples must 
be included in the average. With the current method and computing power, we
cannot gain a definitive answer on the nature of the spin glass phase. 

In the Hybridization Expansion Continuous Time Monte Carlo Solver, we delivered
an impurity solver on the Intel Xeon Phi platform, using the fast update procedures.
We showed that this code is twice as fast than our original CPU implementation, 
and can be easily extend to include more orbitals and complicated interactions.
In collaboration with Roozbeh Karimi and Prof Koppelman, we developed a Krylov 
solver for systems with more orbitals. This impurity can be used for 
calculations on systems with strong interactions, and may be applied in physics 
problems such as high Tc superconductivity.



%%% Local Variables:
%%% mode: latex
%%% TeX-master: "../thesis"
%%% End:
