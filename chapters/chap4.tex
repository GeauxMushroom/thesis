
%%% Local Variables:
%%% mode: latex
%%% TeX-master: "../thesis"
%%% End:



\chapter{Strongly correlated materials}
\section{Introduction}
The term strong correlation refers to the behavior of electrons in solids that 
is not well-described by simple one-electron theories.

\section{Numerical approaches in strongly correlated materials}
Numerical calculation in strongly correlated fermion systems is a major 
challenge in condensed matter physics. In real world, materials consists of 
$10^{23}$ interacting particles, which is impossible to solve at first glance. 
Fortunately, not all the particles contribute to the property of materials. For 
example, In a metal only the electrons close to the Fermi level can be excited 
and contribute ,e.g., to the transport and magnetic properties. In a lattice, 
lattice excitations are few at low T, but they are responsible for inelastic 
neutron scattering. Even though, the remaining problem is still hard. 

Density functional theory (DFT)\cite{PhysRev.136.B864,PhysRev.140.A1133} 
provides a framework to solve the 
electron structure problem. Using this theory, the properties of a many-electron
system can be described by a functional of election density. Combined with 
approximations that address the exchange-correlations such as the local density 
approximation (LDA) \cite{lundqvist1983}, DFT produces satisfactory data that agrees well the 
experiments for many cases. However, despite the success in weakly correlated 
materials, there are still difficulties in applying this method to other 
cases, such as systems with strongly correlation \cite{0953-8984-9-35-010}. 
It seems to be quite
difficult to handle such strong correlation by simple improvements of the 
DFT. A promising direction is to combine other methods which can treat
strong correlation with the DFT method. A popular choice is to employ 
the dynamical mean field theory\citep{RevModPhys.68.13,PhysRevB.45.6479} 
to include the effect from the electron-electron
correlation on top of the single particle dispersion obtained from the DFT. 

%DFT, ab inito

%maybe mention other numerical methods here?

DMFT has been widely studied on a range of strongly correlated systems. 
It is a method well suited for strongly correlated systems, in particular it 
captures the Mott transition, a hallmark of strong correlation, of the Hubbard 
model. In this approach, the solution of the lattice model is mapped to a 
quantum impurity model with self-consistency conditions. 
%intro to impurity
%mention other impurity solvers
A quantum impurity problem describes an atom embedded in a host medium. 
The impurity consists of a set of orbitals with different parameters, populated
with electrons that interacts with each other. The orbitals are hybridized to 
bath orbitals representing the degrees of freedom of the host materials. 
The solution of impurity problem can be obtained in a few different ways. 
We will focus on the different variants of Monte Carlo methods.

A commonly used technique is the Hirsch-Fye method \cite{1986PhRvL..56.2521H}, in which a 
Hubbard-Stratonovich transformation is used to decouple the interaction part,
leading to determinants which give the weights associated with the 
configurations of the auxiliary fields, which are then sampled by a Monte Carlo 
procedure. One issue is that Hirsch-Fye cannot be easily applied to complicated
interaction that include more than just density-density interaction, due to the
lack of simple ansatz to decouple the interacting terms. 
The matrix size scales to the interaction as well as the inverse of 
temperature, which makes the calculation inefficient at low temperatures.
This methods also requires discretization of the imaginary time interval,
which introduce systematic errors, and may not be optimal for multi-orbital case 
with complicated off-diagonal couplings.

The Trotter error in Hirsch-Fye algorithm can be eliminated by using the 
Continuous Time Monte Carlo algorithms. For example, one can solve the problem 
exactly in non-interacting limit, and treat the interaction with a Taylor-series
 expansion. By doing stochastic sampling of diagrams in the weak-coupling 
expansion of partition function, the interaction expansion (CT-INT) algorithm
\cite{2005PhRvB..72c5122R}
provides a discretization error free alternative Hirsch-Fye algorithm.
Still, in the CT-INT algorithm, it is difficult to treat non-Hubbard-type 
interactions. Also the size of the matrix used in the CT-INT method grows
quickly with the interaction, making the calculation very time-consuming at 
very strong interactions.

Another way to treat the impurity problem is the hybridization expansion (CT-HYB)
approach \cite{RevModPhys.83.349, PhysRevB.75.155113, PhysRevB.80.235117,
PhysRevB.74.155107}. 
The fact that the order of expansion decreases with increasing 
interaction makes this method favorable for strong interaction systems. 
The algorithm is also found to work at very low temperatures, and is applicable
to a wider class of impurity models including those with complicated 
off-diagonal couplings, since the local problem is treated exactly. 
